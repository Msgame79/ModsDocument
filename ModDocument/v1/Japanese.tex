%LuaLaTeX
\documentclass[lualatex,a4paper,fontsize=11pt,jafontscale=0.9247,titlepage,oneside]{jlreq}
\ltjsetparameter{jacharrange={-2}}
\usepackage{layout}
\usepackage{luatexja-fontspec}
\setmainfont{NotoSans.ttf}[Instance=Medium]
\setmonofont{NotoSansMono.ttf}[Instance=Medium]
\setmainjfont{NotoSansJP.ttf}[Instance=Medium]
\usepackage[colorlinks=true,allcolors=blue]{hyperref}
\setlength{\hoffset}{-0.4in}
\setlength{\voffset}{-0.6in}
\setlength{\oddsidemargin}{2pt}
\setlength{\topmargin}{3pt}
\setlength{\headheight}{0pt}
\setlength{\headsep}{0pt}
\setlength{\textheight}{760pt}
\setlength{\textwidth}{514pt}
\setlength{\marginparsep}{0pt}
\setlength{\marginparwidth}{0pt}
\setlength{\marginparpush}{4pt}
\setlength{\footskip}{20pt}
\makeatletter
\newcommand*{\themonth}{\two@digits\month}
\newcommand*{\theday}{\two@digits\day}
\makeatother
\renewcommand{\today}{{\the\year}/{\themonth}/{\theday}}
\newcommand{\todayen}{{\themonth}-{\theday}-{\the\year}}
\setcounter{tocdepth}{3}
\newcommand{\mono}[1]{\texttt{#1}}
\newcommand{\parr}{\par\hspace{-1\zw}}
\renewcommand{\jlreqxkanjiskip}{0pt plus 0pt minus 1pt}
\renewcommand{\thesection}{\textbf{\texttt{\arabic{section}.}}}
\renewcommand{\thesubsection}{\textbf{\texttt{\arabic{section}.\arabic{subsection}.}}}
\renewcommand{\thesubsubsection}{\textbf{\texttt{\arabic{section}.\arabic{subsection}.\arabic{subsubsection}.}}}
\renewcommand{\theenumi}{\arabic{enumi}.}
\renewcommand{\theenumii}{\arabic{enumii}.}
\renewcommand{\theenumiii}{\arabic{enumiii}.}
\renewcommand{\theenumiv}{\arabic{enumiv}.}
\renewcommand{\labelenumi}{\texttt{\theenumi}\hspace{0.5em}}
\renewcommand{\labelenumii}{\texttt{\theenumi\theenumii}\hspace{0.5em}}
\renewcommand{\labelenumiii}{\texttt{\theenumi\theenumii\theenumiii}\hspace{0.5em}}
\renewcommand{\labelenumiv}{\texttt{\theenumi\theenumii\theenumiii\theenumiv}\hspace{0.5em}}
\begin{document}
\title{Human Fall Flat Modドキュメント}
\author{著:\;\href{https://www.youtube.com/channel/UCKpnq5EXLCcuPsEGAjyXbsg}{Msgames79}}
\date{初版:\;\today(v1.7.4)}
\maketitle
\mono{This document was compiled with Lua\LaTeX\ from \TeX\ Live 2024.}\par
\mono{Fonts used: Noto Series}
\tableofcontents
\clearpage
\section{概要}
Human Fall Flat用のModをインストール・使用するためのドキュメントです。なお、本書はあくまで情報の提供のみを目的としており、実際に使用することによるすべての責任は読者に存在します。
\section{動作環境}
\begin{itemize}
\item Windows 10もしくは11
\item Human Fall Flat 現行バージョン
\end{itemize}
\section{BepInExのインストール}
Modをインストールするために、\mono{BepInEx}というライブラリをダウンロードし設定する必要があります。
\subsection{Human Fall Flatのゲームフォルダを特定}
\begin{enumerate}
\item \href{steam://open/games/details/477160}{ここ}をクリックしてSteamクライアントのHFFのトップ画面を開く。
\item \label{3.1.2}管理(歯車)→管理→ローカルファイルを閲覧とクリックしてエクスプローラーの新しいウィンドウを開く。
\item \label{3.1.3}\mono{F4}キーを押すとアドレスバーのパスが選択されるので\mono{Ctrl+C}か右クリックでコピーしておく(見えるようにメモ帳に貼り付けても良い)。
\end{enumerate}
\subsection{\mono{BepInEx}のインストール}
\begin{enumerate}
\item \href{https://github.com/BepInEx/BepInEx/releases/latest}{ここ}から\mono{BepInEx\_win\_x86\_(version).zip}をダウンロード。
\item ダウンロードした\mono{BepInEx\_win\_x86\_5.4.23.2.zip}を\hyperref[3.1.3]{\mono{3.1.3}}でコピーしたフォルダに展開。
\item \href{steam://rungameid/477160}{ここ}をクリックしてHuman Fall Flatを起動。
\item Curve Gamesのロゴが出現したら\mono{Alt+F4}などで閉じる。
\item \hyperref[3.1.2]{\mono{3.1.2}}の方法でエクスプローラーを開き、\mono{BepInEx}フォルダ内に\mono{plugins}というフォルダがあれば完了。
\end{enumerate}
\section{plccタイマー}
\subsection{概要}
plccタイマー(\mono{plcc's Timer})は名前の通りplcc(\href{https://github.com/plcc0}{GitHub})(\href{https://space.bilibili.com/111277972}{BiliBili})によって制作されたゲーム内時間(\mono{In Game Time=IGT})を計測するタイマーです。\parr
\mono{BepInEx}ライブラリから呼び出して使用します。
\subsection{インストール}
\begin{enumerate}   
\item \href{https://github.com/Msgame79/hffmods/raw/refs/heads/main/plcc's%20timer/1.7.4/%E6%97%A5%E6%9C%AC%E8%AA%9E.zip}{\mono{ここ}}をクリックして\mono{日本語.zip}をダウンロード。
\item \hyperref[3.1.2]{\mono{3.1.2}}の方法でエクスプローラーを開き、\mono{BepInEx\textbackslash plugins}に移動する。
\item \label{4.2.3}\hyperref[3.1.3]{\mono{3.1.3}}の方法でフォルダへのパスをコピーする。
\item ダウンロードした\mono{日本語.zip}を\hyperref[4.2.3]{\mono{4.2.3}}でコピーしたフォルダに展開。
\end{enumerate}
\subsection{GUI操作}
\mono{Home}キーを押すとメインウィンドウが開きます。
\subsection{メインウィンドウ}
\begin{description}
\item[\textbullet タイマー]タイマーの表示と共通の動作に関係するウィンドウ。
\item[\textbullet 設定]タイマーのカテゴリ別動作に関係するウィンドウ。
\item[\textbullet カスタム]タイマーの見た目を変更するウィンドウ。
\item[\textbullet その他]タイマーと直接関係しないウィンドウ。
\end{description}
\subsubsection{タイマー}
\subsubsection{設定}
\subsubsection{カスタム}
\subsubsection{その他}
\begin{description}
\item[\textbullet 文字サイズ調整]
\item[\textbullet ウィンドウ同期]
\end{description}
\end{document}