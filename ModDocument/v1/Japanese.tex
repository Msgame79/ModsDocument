%LuaLaTeX
\documentclass[lualatex,a4paper,fontsize=11pt,jafontscale=0.9247,titlepage,oneside]{jlreq}
\ltjsetparameter{jacharrange={-2}}
\usepackage{layout}
\usepackage{luatexja-fontspec}
\setmainfont{NotoSans.ttf}[Instance=Medium]
\setmonofont{NotoSansMono.ttf}[Instance=Medium]
\setmainjfont{NotoSansJP.ttf}[Instance=Medium]
\usepackage[colorlinks=true,allcolors=blue]{hyperref}
\setlength{\hoffset}{-0.4in}
\setlength{\voffset}{-0.6in}
\setlength{\oddsidemargin}{2pt}
\setlength{\topmargin}{3pt}
\setlength{\headheight}{0pt}
\setlength{\headsep}{0pt}
\setlength{\textheight}{760pt}
\setlength{\textwidth}{514pt}
\setlength{\marginparsep}{0pt}
\setlength{\marginparwidth}{0pt}
\setlength{\marginparpush}{4pt}
\setlength{\footskip}{20pt}
\makeatletter
\newcommand*{\themonth}{\two@digits\month}
\newcommand*{\theday}{\two@digits\day}
\makeatother
\renewcommand{\today}{{\the\year}/{\themonth}/{\theday}}
\newcommand{\todayen}{{\themonth}-{\theday}-{\the\year}}
\setcounter{tocdepth}{3}
\newcommand{\mono}[1]{\texttt{#1}}
\newcommand{\parr}{\par\hspace{-1\zw}}
\renewcommand{\jlreqxkanjiskip}{0pt plus 0pt minus 1pt}
\renewcommand{\thesection}{\textbf{\texttt{\arabic{section}.}}}
\renewcommand{\thesubsection}{\textbf{\texttt{\arabic{section}.\arabic{subsection}.}}}
\renewcommand{\thesubsubsection}{\textbf{\texttt{\arabic{section}.\arabic{subsection}.\arabic{subsubsection}.}}}
\renewcommand{\theenumi}{\arabic{enumi}.}
\renewcommand{\theenumii}{\arabic{enumii}.}
\renewcommand{\theenumiii}{\arabic{enumiii}.}
\renewcommand{\theenumiv}{\arabic{enumiv}.}
\renewcommand{\labelenumi}{\texttt{\theenumi}\hspace{0.5em}}
\renewcommand{\labelenumii}{\texttt{\theenumi\theenumii}\hspace{0.5em}}
\renewcommand{\labelenumiii}{\texttt{\theenumi\theenumii\theenumiii}\hspace{0.5em}}
\renewcommand{\labelenumiv}{\texttt{\theenumi\theenumii\theenumiii\theenumiv}\hspace{0.5em}}
\begin{document}
\title{Human Fall Flat Modドキュメント}
\author{著:\;\href{https://www.youtube.com/channel/UCKpnq5EXLCcuPsEGAjyXbsg}{Msgames79}}
\date{初版:\;\today}
\maketitle
This document was compiled with Lua\LaTeX\ from \TeX\ Live 2024.\par
Fonts used: Noto Series
\tableofcontents
\clearpage
\section{概要}
Human Fall Flat用のModをインストール・使用するためのドキュメントです。なお、本書はあくまで情報の提供のみを目的としており、実際に使用することによるすべての責任は読者に存在します。
\section{用語と定義}
\begin{description}
\item[Human Fall Flat]
\end{description}
\section{動作環境}
\begin{itemize}
\item Windows 10もしくは11
\item Human Fall Flat 現行バージョン
\end{itemize}
\section{BepInExのインストール}
Modをインストールするために、BepInExというライブラリをダウンロードし設定する必要があります。
\subsection{Human Fall Flatのゲームフォルダを特定}
\begin{enumerate}
\item \href{steam://open/games/details/477160}{ここ}をクリックしてSteamクライアントのHFFのトップ画面を開く。
\item \label{3.1.2}管理(歯車)→管理→ローカルファイルを閲覧とクリックしてエクスプローラーの新しいウィンドウを開く。
\item \label{3.1.3}F4キーを押すとアドレスバーのパスが選択されるのでCtrl+Cか右クリックでコピーしておく(見えるようにメモ帳に貼り付けても良い)。
\end{enumerate}
\subsection{BepInExのインストール}
\begin{enumerate}
\item \href{https://github.com/BepInEx/BepInEx/releases/latest}{ここ}からBepInEx\_win\_x86\_(version).zipをダウンロード。
\item ダウンロードしたBepInEx\_win\_x86\_5.4.23.2.zipを\hyperref[3.1.3]{3.1.3}でコピーしたフォルダに展開。
\item \href{steam://rungameid/477160}{ここ}をクリックしてHuman Fall Flatを起動。
\item Curve Gamesのロゴが出現したらAlt+F4などで閉じる。
\item \hyperref[3.1.2]{3.1.2}の方法でエクスプローラーを開き、BepInExフォルダ内にpluginsというフォルダがあれば完了。
\end{enumerate}
\section{plccタイマー v1.7.4}
\subsection{概要}
plccタイマー(plcc's Timer)は名前の通りplcc(\href{https://github.com/plcc0}{GitHub})(\href{https://space.bilibili.com/111277972}{BiliBili})によって制作されたゲーム内時間(In Game Time=IGT)を計測するタイマーです。\parr
BepInExライブラリから呼び出して使用します。
\subsection{インストール}
\begin{enumerate}   
\item \href{https://github.com/Msgame79/hffmods/raw/refs/heads/main/plcc's%20timer/1.7.4/%E6%97%A5%E6%9C%AC%E8%AA%9E.zip}{ここ}をクリックして日本語.zipをダウンロード。
\item \hyperref[3.1.2]{3.1.2}の方法でエクスプローラーを開き、BepInEx\textbackslash pluginsに移動する。
\item \label{4.2.3}\hyperref[3.1.3]{3.1.3}の方法でフォルダへのパスをコピーする。
\item ダウンロードした日本語.zipを\hyperref[4.2.3]{4.2.3}でコピーしたフォルダに展開。
\end{enumerate}
\subsection{GUI操作}
Homeキーを押すとメインウィンドウが開きます。
\subsection{メインウィンドウ}
\begin{description}
\item[\textbullet タイマー (ラジオボタン)]タイマーの表示と共通の動作に関係するウィンドウ。
\item[\textbullet 設定 (ラジオボタン)]タイマーのカテゴリ別動作に関係するウィンドウ。
\item[\textbullet カスタム (ラジオボタン)]タイマーの見た目を変更するウィンドウ。
\item[\textbullet その他 (ラジオボタン)]タイマーと直接関係しないウィンドウ。
\end{description}
\subsubsection{タイマー}
\begin{description}
\item[\textbullet ]
\item[\textbullet ]
\end{description}
\subsubsection{設定}
\begin{description}
\item[\textbullet ]
\item[\textbullet ]
\end{description}
\subsubsection{カスタム}
\begin{description}
\item[\textbullet ]
\item[\textbullet ]
\end{description}
\subsubsection{その他}
\begin{description}
\item[\textbullet 文字サイズ調整 (チェックボックス)]
\item[\textbullet ウィンドウ同期 (チェックボックス)]
\end{description}
\section{TASMod v1.15.15.2}
\section{Human Mod v1.5.18.4}
\section{Skip Intro Logos v1.0.0.0}
\section{Achievement Tracker v1.2.4 + Achievements Windowed v1.2.1}
\section{cprev v1.1.1}
\section{Grab Count Tracker v1.1.0}
\section{Shell Logger v1.2.0}
\section{Speedrun-Practice Tools v1.0.0.1}
\end{document}